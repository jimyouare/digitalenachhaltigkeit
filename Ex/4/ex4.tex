\documentclass[10pt]{article}

\usepackage[
	a4paper,
	total={150mm, 237mm},
	left=30mm,
	top=30mm,
]{geometry}

% Language
\usepackage[german]{babel}
\usepackage[utf8]{inputenc}

% Fancy Tables
\usepackage{booktabs}

% Math mode
\usepackage{amsfonts, amsmath, amssymb, bm, nccmath}

% For the algorithms
\usepackage[linesnumbered,vlined]{algorithm2e}
\usepackage{algorithmic}

% Better enumeration
\usepackage{enumerate}

% Links
\usepackage{hyperref}
\usepackage{listings}

% Fancy header / footer
\usepackage{fancyhdr}
\pagestyle{fancy}
\fancyhead{}
\fancyfoot{}
\fancyhead[L]{Digitale Nachhaltigkeit \\ Serie \thesection}
\fancyhead[R]{Paco Eggimann\\19-106-103}
\fancyfoot[R]{\thepage}
\renewcommand{\headrulewidth}{1pt}
\renewcommand{\footrulewidth}{1pt}
% Reset header / footer on titlepage
\fancypagestyle{titlepage}{
    \fancyhead{}
    \fancyfoot{}
    \renewcommand{\headrulewidth}{0pt}
    \renewcommand{\footrulewidth}{0pt}
}
\setlength{\headheight}{35pt}

\parindent 0ex

\newcommand{\heading}[2]{\setcounter{section}{#2}\section*{#1 #2}}
\newcommand{\problem}[1]{\subsection{#1}}
\newcommand{\proof}[1]{\subsubsection*{#1}}

\newcommand{\LRarr}{\Leftrightarrow}

\begin{document}

    \heading{Serie}{4}
    \problem{Aufgabe 1}
		Zuerst wollen wir nochmal angeben, was K-Anonymität bedeutet. K- Anonymität steht für die Eigenschaft, das die Kleinste Anzahl der gemeinsamen Einträge in einer Tabelle mindestens K beträgt. Wobei K eine natürliche Zahl ist. In unserrem Fall sollen wir in der Vorgegebenen Excel- Tabelle eine 3 Anonymität erstellen. Die einfachste Methode diese zu erreichen, ist nacheinander die Daten der einzelnen Spalten zu generalisieren bis jeder wert mindestens 3 mal vertreten ist. Im Falle des Datum lässt uns dieses Genau den Monat übrig. Im falle der Matrikel Nr die letzte Ziffer. Bei dem Wohnohrt und der Adresse muss alles verborgen werden um die 3 Anonymität zu erreichen. Weitere Daten zu veröffentlichen führt dazu, dass die 3-Anonymität verletzt wird.

    \problem{Aufgabe 2}
	Bei der Veröffentlichung solcher Daten kommt es darauf an, wie diese Publiziert werden. Aus der Aufgabenstellung entnehme ich, dass nicht anonymisiert wird, um welche Quartiere es sich dabei handelt. Dabei entstehen mehrere Problematiken:
	\begin{enumerate}
	 \item Gerade heutzutage mit dem steigenden Umweltbewustseinm, würde man Wohnhafte in Quartieren mit einem hohen Stromverbrauch an den Pranger stellen. Dies kann so weit gehen, wie das Proteste bewusst in solchen Quartieren abgehalten werden um die Menschen die dort Wohnen zu ``bestrafen'', obwohl diese Unter Umständen nichts dafür können, weil die Infrastrukturen alt sind.
	 \item Ein weiteres Problem entsteht beim Werteverlust der Quartiere, welche einen hohen Stromverbrauch aufweisen. Dadurch machen die Vermieter weniger Geld, was wiederum dazu führt, das mögliche Mängel an der Infrastruktur weniger schnell behoben werden können.
	 \item Nicht zu Letzt verletzen wir die Privatsphäre der Anwohner der Quartiere, mit Telefonbüchern ist es somit sehr einfach zu ermitteln, welche Personen in einem Haushalt Leben, wo ein hoher Stromverbrauch messbar ist. Dies ist entgegengesetzt mit der Privatsphäre und der Datenschutz wird verletzt.
	\end{enumerate}




\end{document}
